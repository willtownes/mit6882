%%%%%%%%%%%%%%%%%%%%%%%%%%%%%%%%%%%%%%%%%
% Beamer Presentation
% LaTeX Template
% Version 1.0 (10/11/12)
%
% This template has been downloaded from:
% http://www.LaTeXTemplates.com
%
% License:
% CC BY-NC-SA 3.0 (http://creativecommons.org/licenses/by-nc-sa/3.0/)
%
%%%%%%%%%%%%%%%%%%%%%%%%%%%%%%%%%%%%%%%%%

%----------------------------------------------------------------------------------------
%	PACKAGES AND THEMES
%----------------------------------------------------------------------------------------

\documentclass{beamer}

\mode<presentation> {

% The Beamer class comes with a number of default slide themes
% which change the colors and layouts of slides. Below this is a list
% of all the themes, uncomment each in turn to see what they look like.

%\usetheme{default}
%\usetheme{AnnArbor}
%\usetheme{Antibes}
%\usetheme{Bergen}
%\usetheme{Berkeley}
\usetheme{Berlin}
%\usetheme{Boadilla}
%\usetheme{CambridgeUS}
%\usetheme{Copenhagen}
%\usetheme{Darmstadt}
%\usetheme{Dresden}
%\usetheme{Frankfurt}
%\usetheme{Goettingen}
%\usetheme{Hannover}
%\usetheme{Ilmenau}
%\usetheme{JuanLesPins}
%\usetheme{Luebeck}
%\usetheme{Madrid}
%\usetheme{Malmoe}
%\usetheme{Marburg}
%\usetheme{Montpellier}
%\usetheme{PaloAlto}
%\usetheme{Pittsburgh}
%\usetheme{Rochester}
%\usetheme{Singapore}
%\usetheme{Szeged}
%\usetheme{Warsaw}

% As well as themes, the Beamer class has a number of color themes
% for any slide theme. Uncomment each of these in turn to see how it
% changes the colors of your current slide theme.

%\usecolortheme{albatross}
\usecolortheme{beaver}
%\usecolortheme{beetle}
%\usecolortheme{crane}
%\usecolortheme{dolphin}
%\usecolortheme{dove}
%\usecolortheme{fly}
%\usecolortheme{lily}
%\usecolortheme{orchid}
%\usecolortheme{rose}
%\usecolortheme{seagull}
%\usecolortheme{seahorse}
%\usecolortheme{whale}
%\usecolortheme{wolverine}

%\setbeamertemplate{footline} % To remove the footer line in all slides uncomment this line
%\setbeamertemplate{footline}[page number] % To replace the footer line in all slides with a simple slide count uncomment this line

%\setbeamertemplate{navigation symbols}{} % To remove the navigation symbols from the bottom of all slides uncomment this line
}

% math definition
\newcommand{\indep}{\mathrel{\text{\scalebox{1.07}{$\perp\mkern-10mu\perp$}}}}


%
\usepackage{graphicx} % Allows including images
\usepackage{booktabs} % Allows the use of \toprule, \midrule and \bottomrule in tables
\usepackage{comment}

%macros from Bob Gray
\usepackage{"./macro/GrandMacros"}
\usepackage{"./macro/Macro_BIO235"}

\usepackage[normalem]{ulem}

% tikz
\usepackage{tikz}
\usetikzlibrary{bayesnet}


%----------------------------------------------------------------------------------------
%	TITLE PAGE
%----------------------------------------------------------------------------------------

\title[HDP-HMM-SLDS]{Hierarchical Dirichlet Process and Switching Linear Dynamical Systems} % The short title appears at the bottom of every slide, the full title is only on the title page

\author{Will Townes, Jeremiah Zhe Liu} % Your name

\date{\today} % Date, can be changed to a custom date

\begin{document}

\begin{frame}
\titlepage % Print the title page as the first slide
\end{frame}

\begin{frame}
\frametitle{Overview} % Table of contents slide, comment this block out to remove it
\tableofcontents % Throughout your presentation, if you choose to use \section{} and \subsection{} commands, these will automatically be printed on this slide as an overview of your presentation
\end{frame}

%----------------------------------------------------------------------------------------
%	PRESENTATION SLIDES
%----------------------------------------------------------------------------------------

%%% Jeremiah Section
\section{Hierarchical Dirichlet Process and Hidden Markov Model}

%%% Begin Will Section here
\section{Switching Linear Dynamical Systems}
\begin{frame}
\frametitle{Linear Dynamical Systems}

For times $t=1\ldots T$, we are given data $y_t\in\mathbb{R}^n$. We assume $y_t$ is a noise observation of a hidden, continuous state $x_t\in\mathbb{R}^d$, with $d\geq n$, and that $x_t$ is a markov chain. The probability model is:
\begin{align*}
x_t|x_{t-1}&\sim\mathcal{N}(Ax_{t-1}+B,\Sigma)\\
y_t|x_t&\sim\mathcal{N}(Cx_t,R)
\end{align*}
We assume $C$ is known.

\end{frame}

\begin{frame}
\frametitle{Switching Linear Dynamical Systems}

We now assume at each time $t$ there is a hidden mode indexed by $z_t\in\{1,\ldots,K\}$ that determines the dynamical regime. The probability model is:
\begin{align*}
x_t|x_{t-1},z_t&\sim\mathcal{N}(A^{(z_t)}x_{t-1}+B^{(z_t)},\Sigma^{(z_t)})\\
y_t|x_t&\sim\mathcal{N}(Cx_t,R)
\end{align*}
In the standard SLDS, $K<\infty$. In the HDP-HMM-SLDS, $K=\infty$.
\end{frame}

\begin{frame}
\frametitle{Algorithms for Inference}
\begin{itemize}
\item Conditional on known dynamical parameters $A,B,\Sigma$ and $z_{1:T}$, the hidden states $x_{1:T}$ are obtained via a \textbf{Kalman sampler}.
\item Conditional on known hidden states $x_{1:T}$, the hidden modes $z_{1:T}$ are obtained via the \textbf{Forward-Backward} (sampling) algorithm.
\item Conditional on known $x_{1:T},z_{1:T}$, the dynamical parameters are obtained via \textbf{multivariate linear regressions} for each of the modes.
\end{itemize}
\end{frame}

\begin{frame}
\begin{figure}
  \centering
  \includegraphics[width = 1\linewidth]{"./plot/lds/01_projectile_known"}
\end{figure}
\end{frame}

\begin{frame}
\begin{figure}
  \centering
  \includegraphics[width = 1\linewidth]{"./plot/lds/02_projectile_unknown"}
\end{figure}
\end{frame}

\begin{frame}
\begin{figure}
  \centering
  \includegraphics[width = 1\linewidth]{"./plot/lds/03_harmonic_known"}
\end{figure}
\end{frame}

\begin{frame}
\begin{figure}
  \centering
  \includegraphics[width = 1\linewidth]{"./plot/lds/04_harmonic_unknown"}
\end{figure}
\end{frame}

\begin{frame}
\frametitle{Lessons Learned}
\begin{enumerate}
\item Kalman sampler great when dynamical parameters known.
\item When dynamical parameters unknown, choice of prior is crucial.
\item ``Uninformative'' hyperparameters didn't work. Need regularization.
\item Non-stationary time series present problems for empirical bayes.
\item Higher dimensional state space is more flexible but may overfit.
\end{enumerate}
\end{frame}

\end{document}